% This is samplepaper.tex, a sample chapter demonstrating the
% LLNCS macro package for Springer Computer Science proceedings;
% Version 2.21 of 2022/01/12
%
\documentclass[runningheads]{llncs}
%
\usepackage[T1]{fontenc}
% T1 fonts will be used to generate the final print and online PDFs,
% so please use T1 fonts in your manuscript whenever possible.
% Other font encondings may result in incorrect characters.
%
\usepackage{graphicx}
% Used for displaying a sample figure. If possible, figure files should
% be included in EPS format.
%
% If you use the hyperref package, please uncomment the following two lines
% to display URLs in blue roman font according to Springer's eBook style:
%\usepackage{color}
%\renewcommand\UrlFont{\color{blue}\rmfamily}
%

\usepackage{colortbl}
\usepackage{color}
\usepackage{pgf}
\definecolor{lightgreen}{rgb}{0.4, 1, 0.4}
\definecolor{lightred}{rgb}{1, 0.31, 0.31}

\begin{document}
%
\title{Contribution Title\thanks{Supported by organization x.}}
%
%\titlerunning{Abbreviated paper title}
% If the paper title is too long for the running head, you can set
% an abbreviated paper title here
%
\author{First Author\inst{1}\orcidID{0000-1111-2222-3333} \and
Second Author\inst{2,3}\orcidID{1111-2222-3333-4444} \and
Third Author\inst{3}\orcidID{2222--3333-4444-5555}}
%
\authorrunning{F. Author et al.}
% First names are abbreviated in the running head.
% If there are more than two authors, 'et al.' is used.
%
\institute{Princeton University, Princeton NJ 08544, USA \and
Springer Heidelberg, Tiergartenstr. 17, 69121 Heidelberg, Germany
\email{lncs@springer.com}\\
\url{http://www.springer.com/gp/computer-science/lncs} \and
ABC Institute, Rupert-Karls-University Heidelberg, Heidelberg, Germany\\
\email{\{abc,lncs\}@uni-heidelberg.de}}
%
\maketitle              % typeset the header of the contribution
%
\begin{abstract}
The abstract should briefly summarize the contents of the paper in
150--250 words.

\keywords{First keyword  \and Second keyword \and Another keyword.}
\end{abstract}
%
%
%
\section{First Section}
\subsection{A Subsection Sample}
Please note that the first paragraph of a section or subsection is
not indented. The first paragraph that follows a table, figure,
equation etc. does not need an indent, either.

Subsequent paragraphs, however, are indented.

\subsubsection{Sample Heading (Third Level)} Only two levels of
headings should be numbered. Lower level headings remain unnumbered;
they are formatted as run-in headings.

\paragraph{Sample Heading (Fourth Level)}
The contribution should contain no more than four levels of headings.

\noindent Displayed equations are centered and set on a separate
line.
\begin{equation}
x + y = z
\end{equation}

\begin{tabular}{l l l l l l r}
    Runs & P\_5 & P\_10 & P\_20 & ndcg & map & num\_rel\_ret \\
    \hline
    \hline
    baseline & 0.512 & 0.510 & 0.452 & 0.362 & 0.164 & 9420 \\
    \hline
    1 & 0.638 & 0.617 & 0.564 & 0.231 & 0.117 & 5207 \\
    2 & 0.616 & 0.624 & 0.578 & 0.386 & 0.187 & 9832 \\
    3 & 0.057 & 0.045 & 0.038 & 0.031 & 0.005 & 1130 \\
    4 & 0.646 & 0.560 & 0.529 & 0.198 & 0.094 & 4354 \\
    5 & 0.646 & 0.581 & 0.569 & 0.251 & 0.126 & 5872 \\
    6 & 0.020 & 0.014 & 0.014 & 0.019 & 0.001 & 643 \\
    7 & 0.020 & 0.014 & 0.014 & 0.019 & 0.001 & 643 \\
    8 & 0.020 & 0.014 & 0.014 & 0.019 & 0.001 & 643 \\
    9 & 0.020 & 0.014 & 0.014 & 0.019 & 0.001 & 643 \\
    10 & 0.020 & 0.014 & 0.014 & 0.019 & 0.001 & 643 \\
    11 & 0.020 & 0.014 & 0.014 & 0.019 & 0.001 & 643 \\
    12 & 0.020 & 0.014 & 0.014 & 0.019 & 0.001 & 643 \\
    13 & 0.020 & 0.014 & 0.014 & 0.019 & 0.001 & 643 \\
    14 & 0.016 & 0.010 & 0.012 & 0.021 & 0.001 & 685 \\
    15 & 0.296 & 0.204 & 0.131 & 0.030 & 0.003 & 608 \\
    16 & 0.328 & 0.226 & 0.137 & 0.036 & 0.004 & 817 \\
    17 & 0.296 & 0.204 & 0.131 & 0.030 & 0.003 & 608 \\
    18 & 0.468 & 0.474 & 0.445 & 0.383 & 0.181 & 9965 \\
    19 & 0.312 & 0.234 & 0.145 & 0.031 & 0.003 & 608 \\
    20 & 0.312 & 0.234 & 0.145 & 0.031 & 0.003 & 608 \\
    21 & 0.312 & 0.234 & 0.145 & 0.031 & 0.003 & 608 \\
    22 & 0.312 & 0.234 & 0.145 & 0.031 & 0.003 & 608 \\
    23 & 0.312 & 0.234 & 0.145 & 0.031 & 0.003 & 608 \\
    24 & 0.272 & 0.178 & 0.112 & 0.028 & 0.003 & 608 \\
    25 & 0.296 & 0.228 & 0.139 & 0.031 & 0.004 & 608 \\
    26 & 0.296 & 0.226 & 0.137 & 0.031 & 0.003 & 612 \\
    27 & 0.688 & 0.678 & 0.652 & 0.397 & 0.199 & 9914 \\
    28 & 0.292 & 0.226 & 0.138 & 0.031 & 0.004 & 612 \\
    29 & 0.020 & 0.010 & 0.012 & 0.021 & 0.001 & 685 \\
    30 & 0.020 & 0.010 & 0.012 & 0.021 & 0.001 & 685 \\
    31 & 0.688 & 0.678 & 0.652 & 0.397 & 0.199 & 9914 \\
    32 & 0.612 & 0.604 & 0.580 & 0.405 & 0.203 & 10276 \\
    33 & 0.612 & 0.604 & 0.580 & 0.405 & 0.203 & 10276 \\
    34 & 0.612 & 0.604 & 0.580 & 0.405 & 0.203 & 10276 \\
    35 & 0.612 & 0.604 & 0.580 & 0.405 & 0.203 & 10276 \\
    36 & 0.692 & 0.672 & 0.638 & 0.403 & 0.202 & 10128 \\
    37 & 0.652 & 0.660 & 0.600 & 0.387 & 0.188 & 9715 \\
    38 & 0.680 & 0.662 & 0.620 & 0.408 & 0.205 & 10261 \\
    39 & 0.680 & 0.662 & 0.620 & 0.408 & 0.206 & 10259 \\
    40 & 0.676 & 0.666 & 0.625 & 0.408 & 0.205 & 10243 \\
    41 & 0.692 & 0.658 & 0.624 & 0.385 & 0.187 & 9669 \\
    42 & 0.664 & 0.656 & 0.611 & 0.405 & 0.201 & 10128 \\
    43 & 0.748 & 0.708 & 0.673 & 0.439 & 0.236 & 10846 \\
    44 & 0.736 & 0.704 & 0.667 & 0.437 & 0.234 & 10809 \\
    45 & 0.740 & 0.716 & 0.682 & 0.442 & 0.241 & 10931 \\
    46 & 0.720 & 0.702 & 0.676 & 0.444 & 0.242 & 10966 \\
    47 & 0.720 & 0.702 & 0.666 & 0.445 & 0.242 & 11015 \\
    48 & 0.740 & 0.722 & 0.680 & 0.442 & 0.240 & 10927 \\
    49 & 0.744 & 0.720 & 0.680 & 0.442 & 0.240 & 10911 \\
    50 & {\cellcolor{lightgreen}} 0.800 & {\cellcolor{lightgreen}} 0.746 & {\cellcolor{lightgreen}} 0.714 & 0.456 & {\cellcolor{lightgreen}} 0.254 & 11134 \\
    51 & 0.764 & 0.726 & 0.700 & 0.455 & 0.246 & 11127 \\
    52 & 0.776 & 0.738 & 0.707 & {\cellcolor{lightgreen}} 0.459 & 0.253 & 11213 \\
    53 & 0.776 & 0.744 & 0.705 & {\cellcolor{lightgreen}} 0.459 & 0.254 & {\cellcolor{lightgreen}} 11230 \\
    \end{tabular}
    
    \begin{equation}
        \exp ((\lambda * \max(0,| fieldvalue \textsubscript{doc} - origin | - offset)) \\
        \lambda = \ln(decay) / scale)  
    \end{equation}
    
    
\begin{theorem}
This is a sample theorem. The run-in heading is set in bold, while
the following text appears in italics. Definitions, lemmas,
propositions, and corollaries are styled the same way.
\end{theorem}
%
% the environments 'definition', 'lemma', 'proposition', 'corollary',
% 'remark', and 'example' are defined in the LLNCS documentclass as well.
%
\begin{proof}
Proofs, examples, and remarks have the initial word in italics,
while the following text appears in normal font.
\end{proof}
\subsubsection{Acknowledgements} Please place your acknowledgments at
the end of the paper, preceded by an unnumbered run-in heading (i.e.
3rd-level heading).

%
% ---- Bibliography ----
%
% BibTeX users should specify bibliography style 'splncs04'.
% References will then be sorted and formatted in the correct style.
%
% \bibliographystyle{splncs04}
% \bibliography{mybibliography}
%
\begin{thebibliography}{8}
\bibitem{ref_article1}
Author, F.: Article title. Journal \textbf{2}(5), 99--110 (2016)

\bibitem{ref_lncs1}
Author, F., Author, S.: Title of a proceedings paper. In: Editor,
F., Editor, S. (eds.) CONFERENCE 2016, LNCS, vol. 9999, pp. 1--13.
Springer, Heidelberg (2016). \doi{10.10007/1234567890}

\bibitem{ref_book1}
Author, F., Author, S., Author, T.: Book title. 2nd edn. Publisher,
Location (1999)

\bibitem{ref_proc1}
Author, A.-B.: Contribution title. In: 9th International Proceedings
on Proceedings, pp. 1--2. Publisher, Location (2010)

\bibitem{ref_url1}
LNCS Homepage, \url{http://www.springer.com/lncs}. Last accessed 4
Oct 2017
\end{thebibliography}
\end{document}
